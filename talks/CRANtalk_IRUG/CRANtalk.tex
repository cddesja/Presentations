\documentclass[xcolor=svgnames]{beamer}\usepackage[]{graphicx}\usepackage[]{color}
%% maxwidth is the original width if it is less than linewidth
%% otherwise use linewidth (to make sure the graphics do not exceed the margin)
\makeatletter
\def\maxwidth{ %
  \ifdim\Gin@nat@width>\linewidth
    \linewidth
  \else
    \Gin@nat@width
  \fi
}
\makeatother

\definecolor{fgcolor}{rgb}{0.345, 0.345, 0.345}
\newcommand{\hlnum}[1]{\textcolor[rgb]{0.686,0.059,0.569}{#1}}%
\newcommand{\hlstr}[1]{\textcolor[rgb]{0.192,0.494,0.8}{#1}}%
\newcommand{\hlcom}[1]{\textcolor[rgb]{0.678,0.584,0.686}{\textit{#1}}}%
\newcommand{\hlopt}[1]{\textcolor[rgb]{0,0,0}{#1}}%
\newcommand{\hlstd}[1]{\textcolor[rgb]{0.345,0.345,0.345}{#1}}%
\newcommand{\hlkwa}[1]{\textcolor[rgb]{0.161,0.373,0.58}{\textbf{#1}}}%
\newcommand{\hlkwb}[1]{\textcolor[rgb]{0.69,0.353,0.396}{#1}}%
\newcommand{\hlkwc}[1]{\textcolor[rgb]{0.333,0.667,0.333}{#1}}%
\newcommand{\hlkwd}[1]{\textcolor[rgb]{0.737,0.353,0.396}{\textbf{#1}}}%

\usepackage{framed}
\makeatletter
\newenvironment{kframe}{%
 \def\at@end@of@kframe{}%
 \ifinner\ifhmode%
  \def\at@end@of@kframe{\end{minipage}}%
  \begin{minipage}{\columnwidth}%
 \fi\fi%
 \def\FrameCommand##1{\hskip\@totalleftmargin \hskip-\fboxsep
 \colorbox{shadecolor}{##1}\hskip-\fboxsep
     % There is no \\@totalrightmargin, so:
     \hskip-\linewidth \hskip-\@totalleftmargin \hskip\columnwidth}%
 \MakeFramed {\advance\hsize-\width
   \@totalleftmargin\z@ \linewidth\hsize
   \@setminipage}}%
 {\par\unskip\endMakeFramed%
 \at@end@of@kframe}
\makeatother

\definecolor{shadecolor}{rgb}{.97, .97, .97}
\definecolor{messagecolor}{rgb}{0, 0, 0}
\definecolor{warningcolor}{rgb}{1, 0, 1}
\definecolor{errorcolor}{rgb}{1, 0, 0}
\newenvironment{knitrout}{}{} % an empty environment to be redefined in TeX

\usepackage{alltt} 
%\usecolortheme[named=maroon]{structure} 
%\usetheme[height=7mm]{Madrid} 
%\usepackage{hyperref}
%\beamertemplatenavigationsymbolsempty
\usepackage{pdfpages}
%\hypersetup{
%colorlinks=true}

% \usepackage{beamerthemesplit} // Activate for custom appearance

\title{CRAN workshop}
\author{Christopher David Desjardins
\url{http://www.github.com/cddesjardins}}
\date{6 March 2014}
\IfFileExists{upquote.sty}{\usepackage{upquote}}{}
\begin{document}
\SweaveOpts{concordance=TRUE}
\frame{\titlepage}
  
\frame{
\frametitle{Overview}
\tableofcontents}

\section{Brief introduction to programming in R}


\begin{frame}
  \frametitle{Objects in R (based on Wickham, 2011)}
  \begin{itemize}
  \item The R language has three objects systems, S3, S4, and R5.
  \item S3
    \begin{itemize}
      \item Not a formal class
      \item S3 has been around since R started.
      \item S3 is really just a naming convention and a bunch of methods.
        \item Very easy to use and most common in R (only OO in base and stats)
   \end{itemize}
  \item S4
    \begin{itemize}
      \item Similar to S3 but newer and formal. Rather difficult.
        \item \texttt{setClass()}, \texttt{setGeneric()}, and \texttt{setMethod()} define classes, generic functions, and methods. Include slots @ (for example, see lmerMod)
    \end{itemize}
    \item R5
      \begin{itemize}
        \item Suited for simulations that model complex states and GUI. Components that need mutable states
      \end{itemize}
      \item Can use \texttt{pryr::otype(foo)} to determine object type
  \end{itemize}
\end{frame}

\begin{frame}[fragile]
  \frametitle{S3 objects}
  \scriptsize{
  \begin{itemize}
  \item Methods for objects of a particular class are called by \texttt{method.class()}
  \item Methods include \texttt{summary()}, \texttt{anova()}, \texttt{print()}, etc.
  \item Classes include \texttt{lm}, \texttt{glm}, \texttt{factor}, and so on.
  \end{itemize}
  }
\tiny{
\begin{knitrout}
\definecolor{shadecolor}{rgb}{0.969, 0.969, 0.969}\color{fgcolor}\begin{kframe}
\begin{alltt}
\hlkwd{library}\hlstd{(profileR)}
\end{alltt}


{\ttfamily\noindent\bfseries\color{errorcolor}{\#\# Error in library(profileR): there is no package called 'profileR'}}\begin{alltt}
\hlkwd{ls}\hlstd{(package}\hlopt{:}\hlstd{profileR)}
\end{alltt}


{\ttfamily\noindent\color{warningcolor}{\#\# Warning in ls(package:profileR): 'package:profileR' converted to character string}}

{\ttfamily\noindent\bfseries\color{errorcolor}{\#\# Error in as.environment(pos): no item called "{}package:profileR"{} on the search list}}\begin{alltt}
\hlkwd{methods}\hlstd{(}\hlkwc{class} \hlstd{=} \hlstr{"critpat"}\hlstd{)}
\end{alltt}
\begin{verbatim}
## no methods found
\end{verbatim}
\begin{alltt}
\hlkwd{methods}\hlstd{(summary)}
\end{alltt}
\begin{verbatim}
##  [1] summary.aov                    summary.aovlist*              
##  [3] summary.aspell*                summary.check_packages_in_dir*
##  [5] summary.connection             summary.data.frame            
##  [7] summary.Date                   summary.default               
##  [9] summary.ecdf*                  summary.factor                
## [11] summary.glm                    summary.infl*                 
## [13] summary.lm                     summary.loess*                
## [15] summary.manova                 summary.matrix                
## [17] summary.mlm*                   summary.nls*                  
## [19] summary.packageStatus*         summary.PDF_Dictionary*       
## [21] summary.PDF_Stream*            summary.POSIXct               
## [23] summary.POSIXlt                summary.ppr*                  
## [25] summary.prcomp*                summary.princomp*             
## [27] summary.proc_time              summary.srcfile               
## [29] summary.srcref                 summary.stepfun               
## [31] summary.stl*                   summary.table                 
## [33] summary.tukeysmooth*          
## see '?methods' for accessing help and source code
\end{verbatim}
\end{kframe}
\end{knitrout}
}
\end{frame}

\begin{frame}[fragile]
  \frametitle{More with S3}
\begin{knitrout}
\definecolor{shadecolor}{rgb}{0.969, 0.969, 0.969}\color{fgcolor}\begin{kframe}


{\ttfamily\noindent\bfseries\color{errorcolor}{\#\# Error in library(pryr): there is no package called 'pryr'}}\end{kframe}
\end{knitrout}
\begin{tiny}

\begin{knitrout}
\definecolor{shadecolor}{rgb}{0.969, 0.969, 0.969}\color{fgcolor}\begin{kframe}
\begin{alltt}
\hlstd{x} \hlkwb{<-} \hlkwd{rbinom}\hlstd{(}\hlkwc{n} \hlstd{=} \hlnum{1000}\hlstd{,} \hlkwc{size} \hlstd{=} \hlnum{1}\hlstd{,} \hlkwc{prob} \hlstd{=} \hlnum{.4}\hlstd{)}
\hlstd{y} \hlkwb{<-} \hlkwd{rnorm}\hlstd{(}\hlkwc{n}\hlstd{=} \hlnum{1000}\hlstd{,}\hlkwc{mean}\hlstd{=}\hlnum{1.4} \hlopt{+} \hlnum{.3}\hlopt{*}\hlstd{x)}
\hlstd{mod1} \hlkwb{<-} \hlkwd{lm}\hlstd{(y}\hlopt{~}\hlstd{x)}
\hlkwd{class}\hlstd{(mod1)}
\end{alltt}
\begin{verbatim}
## [1] "lm"
\end{verbatim}
\begin{alltt}
\hlkwd{otype}\hlstd{(mod1)}
\end{alltt}


{\ttfamily\noindent\bfseries\color{errorcolor}{\#\# Error in eval(expr, envir, enclos): could not find function "{}otype"{}}}\begin{alltt}
\hlcom{# Can call directly, not advised!}
\hlcom{# print.lm(mod1)}

\hlcom{# Need for Sweave}
\hlkwd{cat}\hlstd{(}\hlkwd{try}\hlstd{(}\hlkwd{summary.factor}\hlstd{(mod1)))}
\end{alltt}
\begin{verbatim}
## Error in table(object) : all arguments must have the same length
\end{verbatim}
\begin{alltt}
\hlstd{stats}\hlopt{:::}\hlkwd{extractAIC.lm}\hlstd{(mod1)}
\end{alltt}
\begin{verbatim}
## [1]  2.0000 36.3801
\end{verbatim}
\end{kframe}
\end{knitrout}
\end{tiny}
\end{frame}

\begin{frame}
  \frametitle{S4 objects}
  \begin{itemize}
    \item These are much more complicated
      \item Best to learn S3 first
        \item To learn about S4: \url{http://adv-r.had.co.nz/S4.html}
          \item A great book for learning about the nuts and bolts of R is: \url{adv-r.had.co.nz/}
            \item \texttt{lme4} and \texttt{Matrix} are S4 packages
      
  \end{itemize}
\end{frame}

\section{Developing an R package}

\begin{frame}
\frametitle{Prepping for an R package}
\begin{itemize}
  \item Pull a package off CRAN that you're interested
    \begin{itemize}
      \item Suggestion: Package by Doug Bates, Hadley Wickham, or John Fox
    \end{itemize}
  \item Extract the tarball
  \item Examine the contents
  \item Use package.skeleton() when you have all the data and R functions you want to include in a package in a clean R environment
    \end{itemize}
\end{frame}


\begin{frame}[fragile]
\frametitle{What is in an R package?}
\begin{itemize}
\item DESCRIPTION - description of the package, author(s), and license.
\item A \texttt{man$/$} subdirectory of documentation for each R function.
\item An  \texttt{R$/$} subdirectory of the actual R code.
\item A \texttt{data$/$} subdirectory of datasets.
\item Maybe a \texttt{src$/$} containing C, C++, or Fortran code.
\item \texttt{tests$/$} for validation tests.
\item \texttt{exec$/$} for other executables (eg Perl or Java).
\item \texttt{inst$/$} for miscellaneous other stuff.
\item \texttt{configure} script
\item CHANGELOG - description of the changes
\item NEWS - information about changes in the package
\item NAMESPACE -  What variables in the package should be exported to make them available to package users, and which variables should be imported from other packages
\item A Vignette?
\end{itemize}
\end{frame}

\begin{frame}[fragile]
  \frametitle{package.skeleton()}
\begin{knitrout}
\definecolor{shadecolor}{rgb}{0.969, 0.969, 0.969}\color{fgcolor}\begin{kframe}
\begin{alltt}
\hlstd{trim} \hlkwb{<-} \hlkwa{function}\hlstd{(}\hlkwc{x}\hlstd{,}\hlkwc{tr}\hlstd{=}\hlnum{.1}\hlstd{)\{}
    \hlstd{y}\hlkwb{=}\hlkwd{sort}\hlstd{(x)}
    \hlstd{n}\hlkwb{=}\hlkwd{length}\hlstd{(x)}
    \hlstd{qlow}\hlkwb{=}\hlkwd{quantile}\hlstd{(y,}\hlkwc{probs}\hlstd{=tr,}\hlkwc{na.rm}\hlstd{=T)}
    \hlstd{qhigh}\hlkwb{=}\hlkwd{quantile}\hlstd{(y,}\hlkwc{probs}\hlstd{=}\hlnum{1}\hlopt{-}\hlstd{tr,}\hlkwc{na.rm}\hlstd{=T)}
    \hlstd{y}\hlkwb{=}\hlkwd{subset}\hlstd{(y,y} \hlopt{>} \hlstd{qlow} \hlopt{&} \hlstd{y} \hlopt{<} \hlstd{qhigh)}
    \hlstd{trim}\hlkwb{=}\hlkwd{mean}\hlstd{(y)}
    \hlstd{output} \hlkwb{<-} \hlkwd{list}\hlstd{(}\hlkwc{samp.size} \hlstd{= n,} \hlkwc{untrimmed} \hlstd{=} \hlkwd{mean}\hlstd{(x),}
                   \hlkwc{adj.mean} \hlstd{= trim,} \hlkwc{trim.value} \hlstd{= tr)}
    \hlkwd{class}\hlstd{(output)} \hlkwb{<-} \hlstr{"funMeans"}
    \hlkwd{return}\hlstd{(output)}
\hlstd{\}}
\hlkwd{set.seed}\hlstd{(}\hlnum{2351234}\hlstd{)}
\hlstd{trim.data} \hlkwb{<-} \hlkwd{rnorm}\hlstd{(}\hlkwc{n} \hlstd{=} \hlnum{10}\hlstd{,} \hlkwc{mean} \hlstd{=} \hlnum{5}\hlstd{,} \hlkwc{sd} \hlstd{=} \hlnum{25}\hlstd{)}
\hlcom{# Create a package called funMeans}
\hlcom{#package.skeleton(name = "funMeans")  }
\end{alltt}
\end{kframe}
\end{knitrout}

\end{frame}

\begin{frame}[fragile]
  \frametitle{Read-and-delete-me}
  \begin{verbatim}
* Edit the help file skeletons in 'man', possibly
 combining help files for multiple functions.
* Edit the exports in 'NAMESPACE', and add necessary
 imports.
* Put any C/C++/Fortran code in 'src'.
* If you have compiled code, add a useDynLib() directive
 to 'NAMESPACE'.
* Run R CMD build to build the package tarball.
* Run R CMD check to check the package tarball.

Read "Writing R Extensions" for more information.
\end{verbatim}
\end{frame}

\begin{frame}[fragile]
\frametitle{DESCRIPTION}
\begin{verbatim}
Package: funMeans
Type: Package
Title: What the package does (short line)
Version: 1.0
Date: 2014-03-05
Author: Who wrote it
Maintainer: Who to complain to <yourfault@somewhere.net>
Description: More about what it does (maybe more than one
line)
License: What license is it under?
\end{verbatim}
\end{frame}

\begin{frame}[fragile]
\frametitle{DESCRIPTION}
\begin{footnotesize}
\begin{verbatim}
Package: funMeans
Type: Package
Title: Functions of Means
Version: 0.0.1
Date: 2014-03-05
Author: Christopher David Desjardins <cddesjardins@gmail.com>
Maintainer: Christopher David Desjardins <cddesjardins@gmail.com>
Description: This package will report different means.
License: GPL (>= 2)
\end{verbatim}
\end{footnotesize}
\end{frame}

\begin{frame}[fragile]
  \frametitle{Licensing your package}
  \begin{itemize}
  \item There are several licenses to choose from
    \begin{itemize}
    \item GNU GPL
      \begin{itemize}
      \item Free to share, free to modify, free to copy, free to use, free to study
      \item Freest license ... with a catch!
      \item All derivative work must be licensed under the GPL - copyleft
      \item You can sell GPL software provided you provide the sources upon request or package them with the binary. Then that person could freely distribute the source code ...  
      \end{itemize}
\item MIT
  \begin{itemize}
  \item Permits reuse of your code within proprietary software provided MIT license is included
  \end{itemize}

    \end{itemize}

  \end{itemize}
\end{frame}

\begin{frame}
  \frametitle{/R subdirectory}
  \begin{itemize}
  \item Can safely delete packageName-internal.R
  \item trim.R is the script that contains the actual R code
  \item Let's create a summary method for class \texttt{funcMeans}! 
  \end{itemize}
\end{frame}

\begin{frame}[fragile]
  \frametitle{summary.funMeans}
\begin{knitrout}
\definecolor{shadecolor}{rgb}{0.969, 0.969, 0.969}\color{fgcolor}\begin{kframe}
\begin{alltt}
  \hlstd{summary.funMeans} \hlkwb{<-} \hlkwa{function}\hlstd{(}\hlkwc{object}\hlstd{,} \hlkwc{...}\hlstd{)\{}
      \hlkwd{cat}\hlstd{(}\hlstr{"\textbackslash{}n Adjusted Mean\textbackslash{}n"}\hlstd{)}
      \hlkwd{print}\hlstd{(object}\hlopt{$}\hlstd{adj.mean)}
      \hlkwd{cat}\hlstd{(}\hlstr{"\textbackslash{}n Trimming Value - "}\hlstd{, object}\hlopt{$}\hlstd{trim.value)}
      \hlkwd{cat}\hlstd{(}\hlstr{"\textbackslash{}n Untrimmed Mean - "}\hlstd{, object}\hlopt{$}\hlstd{untrimmed)}
      \hlkwd{cat}\hlstd{(}\hlstr{"\textbackslash{}n Sample size - "}\hlstd{, object}\hlopt{$}\hlstd{samp.size,}\hlstr{"\textbackslash{}n"}\hlstd{)}
  \hlstd{\}}
\hlstd{funMean1} \hlkwb{<-} \hlkwd{trim}\hlstd{(trim.data)}
\hlkwd{summary}\hlstd{(funMean1)}
\end{alltt}
\begin{verbatim}
## 
##  Adjusted Mean
## [1] 10.86191
## 
##  Trimming Value -  0.1
##  Untrimmed Mean -  7.757717
##  Sample size -  10
\end{verbatim}
\end{kframe}
\end{knitrout}
\end{frame}

\begin{frame}[fragile]
\frametitle{One more mean function}
\begin{scriptsize}
\begin{knitrout}
\definecolor{shadecolor}{rgb}{0.969, 0.969, 0.969}\color{fgcolor}\begin{kframe}
\begin{alltt}
\hlstd{win}\hlkwb{<-}\hlkwa{function}\hlstd{(}\hlkwc{x}\hlstd{,}\hlkwc{tr}\hlstd{=}\hlnum{.2}\hlstd{)\{}
   \hlstd{y}\hlkwb{<-}\hlkwd{sort}\hlstd{(x)}
   \hlstd{n}\hlkwb{<-}\hlkwd{length}\hlstd{(x)}
   \hlstd{ibot}\hlkwb{<-}\hlkwd{floor}\hlstd{(tr}\hlopt{*}\hlstd{n)}\hlopt{+}\hlnum{1}
   \hlstd{itop}\hlkwb{<-}\hlkwd{length}\hlstd{(x)}\hlopt{-}\hlstd{ibot}\hlopt{+}\hlnum{1}
   \hlstd{xbot}\hlkwb{<-}\hlstd{y[ibot]}
   \hlstd{xtop}\hlkwb{<-}\hlstd{y[itop]}
   \hlstd{y}\hlkwb{<-}\hlkwd{ifelse}\hlstd{(y}\hlopt{<=}\hlstd{xbot,xbot,y)}
   \hlstd{y}\hlkwb{<-}\hlkwd{ifelse}\hlstd{(y}\hlopt{>=}\hlstd{xtop,xtop,y)}
   \hlstd{win}\hlkwb{<-}\hlkwd{mean}\hlstd{(y)}
   \hlstd{output} \hlkwb{<-} \hlkwd{list}\hlstd{(}\hlkwc{samp.size} \hlstd{= n,} \hlkwc{untrimmed} \hlstd{=} \hlkwd{mean}\hlstd{(x),}
                  \hlkwc{adj.mean} \hlstd{= win,} \hlkwc{trim.value} \hlstd{= tr)}
    \hlkwd{class}\hlstd{(output)} \hlkwb{<-} \hlstr{"funMeans"}
    \hlkwd{return}\hlstd{(output)}
\hlstd{\}}
\hlstd{funMean2} \hlkwb{<-} \hlkwd{win}\hlstd{(trim.data)}
\hlkwd{summary}\hlstd{(funMean2)}
\end{alltt}
\begin{verbatim}
## 
##  Adjusted Mean
## [1] 10.53635
## 
##  Trimming Value -  0.2
##  Untrimmed Mean -  7.757717
##  Sample size -  10
\end{verbatim}
\end{kframe}
\end{knitrout}
\end{scriptsize}
\end{frame}

\begin{frame}
  \frametitle{man/ subdirectory}
  \begin{itemize}
  \item This contains all the manpages
  \item This is time consuming work ... 
    \item Apparently \texttt{roxygen2} can help (untested)
      \item Syntax similar to \LaTeX
  \end{itemize}
  
\end{frame}

\begin{frame}
  \frametitle{Vignette}
  \begin{itemize}
  \item Why you should create a vignette.
    \begin{itemize}
    \item It demonstrates how to use your package
    \item It allows you to clearly articulate your thinking
    \item It can develop into an easy publication for JSS or other similar outlets

    \end{itemize}
  \item You can include either an .Rnw, .tex, or a .pdf document
  \item These should be placed in the vignettes/ subdirectory
        \item Check out \texttt{vignette()}
  \end{itemize}
\end{frame}

\section{Building the package}

\begin{frame}[fragile]
\frametitle{Building the tarball}
   \begin{itemize}
  \item There are various ways to do this
     \begin{itemize}
     \item Rstudio - Use Build \& Reload button
     \item Terminal - Preferred method (because I use Emacs)
     \end{itemize}
  \end{itemize}

  \begin{scriptsize}
  \begin{verbatim}
chris@debian:~/github/$ R CMD build funMeans/
* checking for file 'funMeans/DESCRIPTION' ... OK
* preparing 'funMeans':
* checking DESCRIPTION meta-information ... OK
* checking for LF line-endings in source and make files
* checking for empty or unneeded directories
* looking to see if a 'data/datalist' file should be added
* building 'funMeans_0.0.1.tar.gz'

 \end{verbatim}
\end{scriptsize}
\end{frame}

\begin{frame}[fragile]
\frametitle{Checking if it's OK for CRAN}  
\begin{itemize}
  \item Again you can do this from Rstudio or a Terminal
\end{itemize}
    \begin{scriptsize}
\begin{verbatim}
  chris@debian:~/github/$ R CMD check --as-cran funMeans_0.0.1.tar.gz
* using log directory '/home/chris/github/funMeans.Rcheck'
* using R version 3.0.2 (2013-09-25)
* using platform: x86_64-pc-linux-gnu (64-bit)
 ...
* checking CRAN incoming feasibility ... NOTE
Maintainer: 'Christopher David Desjardins <cddesjardins@gmail.com>'
New submission
...
* checking PDF version of manual ... OK

NOTE: There was 1 note.
\end{verbatim}
\end{scriptsize}
\begin{itemize}
  \item This is the only Note you can ignore!
    \item Any warnings and notes must be dealt with before you submit to CRAN.
      \item Most messages are insightful!
\end{itemize}
\end{frame}

\begin{frame}
  \frametitle{Submitting to CRAN}
  \begin{itemize}
  \item To submit to CRAN follow the instructions here: \url{http://cran.r-project.org/submit.html}
  \item Send an email to CRAN@R-project.org from the maintainer address listed in the package using the subject line 'CRAN submission PACKAGE VERSION', where PACKAGE and VERSION are the package name and version, respectively. 
  \item For a new submission, confirm in your email that you have read and agree to CRANs policies.
    \item Usually in less than 48 hours your package will be on CRAN and binaries will be built for Windows and Macs.
  \end{itemize}
\end{frame}

\section{Github and prompting your package}

\begin{frame}[fragile]
  \frametitle{Github}
  \begin{itemize}
  \item If you are releasing your software with an open-source license (you are right), then consider developing the package on Github.
    \item This code is dumped into a funMeans repository
      \item \url{https://github.com/cddesjardins/funMeans}
        \begin{itemize}
        \item \texttt{git clone https://github.com/cddesjardins/funMeans.git}
        \end{itemize}
  \end{itemize}    
 
\begin{knitrout}
\definecolor{shadecolor}{rgb}{0.969, 0.969, 0.969}\color{fgcolor}\begin{kframe}
\begin{alltt}
\hlkwd{library}\hlstd{(devtools)}
\hlkwd{install_github}\hlstd{(}\hlkwc{username}\hlstd{=}\hlstr{"cddesjardins"}\hlstd{,}\hlkwc{repo}\hlstd{=}\hlstr{"funMeans"}\hlstd{)}
\end{alltt}
\end{kframe}
\end{knitrout}
\end{frame}

\begin{frame}
  \frametitle{Github}
  \begin{itemize}
    \item Great way to share you code
    \item Git is branching and merging of code is good (supposedly)
      \item Very quick and easy to get up and running
      \item Can add other languages too.
    \end{itemize}
\end{frame}

\begin{frame}
  \frametitle{How will someone use your code?}
  \begin{itemize}
  \item Write and publish your vignette
  \item Try and get your package included in a CRAN Task View
  \item Create a website for it (Github can do this)
  \item Promote it in your signature in your email
  \item Contribute on Stack Overflow or the R-mailing list 
  \end{itemize}

  \end{frame}

\begin{frame}
\frametitle{R Users Group}
\begin{itemize}
\item Interested in more R talk?
\item Contact annahelgajons AT gmail DOT com to be added to the mailing list. 
\end{itemize}
\end{frame}
  
\end{document}
